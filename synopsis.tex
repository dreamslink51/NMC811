% \section*{Synopsis}
% Rechargeable lithium ion batteries (LIBs) are currently the most promising power source for a wide range of technologies ranging from portable consumer electronics, automotive vehicles, and mobility equipment, to large scale renewable grid energy storage applications \cite{}.  However, for such applications, further advancements in LIBs are essential in order to achieve higher energy densities, longer charge-discharge life cycles, lower costs, and higher stabilities to meet the requirements of these technologies.  Current LIB systems are complicated by a variety of physico-chemical processes that take place within dynamic microstructures of the battery material (anode and cathode), during cell cycling. 
% Extensive research efforts are underway to improve existing battery materials, developing a better understanding of the degradation mechanisms in LIBs. This review summaries multi-scale models and experimental techniques essential for understanding and optimizing battery performance.

% Layered lithium transition metal oxides (e.g. Li$M$O$_{2}$, where $M$ = Co, Ni, Mn, etc.) are considered as the first generation cathode materials in commercial LIBs, which although possessing a theoretical specific capacity of 270 mAh/g show a limited practical capacity below 200 mAh/g \cite{Myung2017}. LiCoO$_{2}$ is problematic due to high cost of cobalt and the ethical questions regarding the geopolitical issues \cite{mo_impact_2018}. Other oxides such as Li$_{x}$NiO$_{2}$ show capacity fade and poor safety \cite{min_comparative_2016}, while Li$_{x}$Mn$_{2}$O$_{4}$ shows low capacity \cite{tian_performance_2018} as cathode materials.

% A promising alternative is to partially replace Co in LiCoO$_{2}$ with Ni and Mn to obtain layered LiNi$_{x}$Mn$_{y}$Co$_{z}$O$_{2}$ ($x+y+z=1$) system, commonly abbreviated as NMC \cite{rozier_review_2015}. These systems show improved electrochemical performance while reducing the material cost \cite{ohzuku_layered_2001}. The composition of Ni, Mn, and Co can be tuned to obtain various NMC structures. For the application in long-range electric vehicles companies are now switching to LiNi$_{0.8}$Mn$_{0.1}$Co$_{0.1}$O$_{2}$ (NMC811) for the commercial production [ref1]\cite{}, as use of nickel-rich cathodes increases energy density, while reducing cobalt content. Ni-rich NMCs are considered to be the cathode of choice for future all-solid-state LIBs \cite{Myung2017}. In the range of studies showcased here, we have employed combined multi-scale modelling and experimental approaches, focusing on the Ni-rich NMC811 system to investigate the structure and properties relationship of NMC cathodes.

% We have investigated the influence of the valence states of transition metals (TMs) on the stability and structure-property relationships of NMC811 using various first principles electronic structure methods. \re{[Rana-Paper]\cite{rana}} Four strategies were employed to quantify the oxidation states of the transition metals: comparison of magnetic moments, analysis of Wannier functions, calculation of projected density of states (PDOS), and analysis of the Jahn-Teller distortion effect.\cite{yang2019highly,yang2020chemical} Based on this data, the lowest energy structures comprised 1Ni$^{2+}$, 7Ni$^{3+}$, 1Mn$^{4+}$, and 1Co$^{3+}$ in the transition metal layer. A disproportionation from 2Ni$^{3+}$ to Ni$^{2+}$ + Ni$^{4+}$ is also evident in the pristine NMC811 structure. These findings are important for the improvement of electrochemical behaviour of cathodes as well as for the understanding of structural degradation of LIBs during cycling.

% On the dynamic properties, phonon dispersion, phonon density of states, and thermal properties have been reported for Li$M$O ($M$ = Co, Ni, Mn) and various NMC compositions. The oxidation and spin states of the transition metal cations are found to strongly influence the structural dynamics.  The low phonon frequency phonon branches that are mainly contributed by the TM elements move toward a higher-frequency regime with increasing Ni content. This indicates the strengthening of the TM-O bonds in the Ni-rich compounds, which is consistent with the general decreasing lattice constant in the c-axis \cite{sun_electronic_2017}. The thermal conductivity of NMC is suppressed with decreasing Co content and increasing Ni/Mn content. NMC811 was predicted to have a thermal conductivity of 10.3 Wm$^{-1}$K$^{-1}$ (see \ref{figure_kappa})and $\sim$4 times lower than that of LiCoO$_{2}$ \cite{yang2019highly,yang2020chemical}. Additionally, NMC811 has the challenges of having low structural stability, thermal stability, and elastic modulus.

% Li-ion migration during charge/discharge processes is an important phenomenon that governs the battery operation. For this reason we have employed first principles DFT and Classical Molecular Dynamics (MD) approaches to study the local Li hop mechanisms and long-range Li diffusion mechanisms. Our continuous work untangles the complex structural and electrochemical influence on forming classical MD interatomic potentials and the considerations required to conduct Li diffusion studies at different states of charge. This work has led to the creation of a new potential fitting code and a wide discussion on whether the complex NMC structure is possible to represent using a classical model \cite{Morgan2020BuckFit}.

% The lifetime of the NMC811 cathode material is limited due to faster capacity fading, impedance rise, and loss of active material. These occur due to various degradation mechanisms such as mechanical degradation, passivation layer formation, oxygen evolution, transition metal dissolution, and growth of the cathode solid electrolyte interphase (CSEI) \cite{Li2019degra}. In a study showcased here [Abir-Paper] \cite{}, continuum models have been developed to understand these degradation phenomena and to solve these issues by informing material design. Concurrently the thermodynamic and kinetic behaviour of NMC811 have been studied using the Doyle-Fuller-Newman (DFN) model \cite{hui2020param} to optimize the performance of LIBs. The advantage of continuum-scale models is they bridge atomistic level understandings with the experimental observations at the cell level during cycling and calendar ageing. In this regard, the modelling data will be further complemented with experimental measurements on various degradative processes affecting NMC811 application in LIBs, including the above-mentioned degradation mechanisms.

% Increasing commercialization of high-Ni layered lithium nickel manganese cobalt oxide (LiNi$_x$Mn$_y$Co$_z$O$_2$, NMC) positive electrodes (PEs) necessitates degradation studies on  these materials. The study of degradation mechanisms is important to make knowledge available for the battery manufacturers to produce Li-ion batteries with higher specific capacity at a significantly lower cost. \cite{du2020battery,jaguemont2016comprehensive} Among the other variants of high-Ni NMC structures, NMC811 PEs provide higher reversible capacity because a higher number of Li-ions can be extracted during operation within the same voltage window. \cite{andre2015future,noh2013comparison} However, NMC811 structures are not stable when an exceedingly large fraction of Li-ion is removed during charging of the cell and the structures lose their retained capacity easily. \cite{gabrisch2008transmission,liu2015nickel}

